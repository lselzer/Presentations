\documentclass[11pt, a4paper]{article}
\usepackage{amsfonts, amsmath, hanging, hyperref, parskip, times}
\usepackage[numbers]{natbib}
\usepackage[pdftex]{graphicx}
\hypersetup{
  colorlinks,
  linkcolor=blue,
  urlcolor=blue,
  citecolor=blue
}

\let\section=\subsubsection
\newcommand{\pkg}[1]{{\normalfont\fontseries{b}\selectfont #1}} 
\let\proglang=\textit
\let\code=\texttt 
\renewcommand{\title}[1]{\begin{center}{\bf \LARGE #1}\end{center}}
\newcommand{\affiliations}{\footnotesize\centering}
\newcommand{\keywords}{\paragraph{Keywords:}}

\setlength{\topmargin}{-15mm}
\setlength{\oddsidemargin}{-2mm}
\setlength{\textwidth}{165mm}
\setlength{\textheight}{250mm}

\begin{document}
\pagestyle{empty}

\title{\pkg{rgeos}: spatial geometry predicates and topology operations in \proglang{R}}


\begin{center}
  {\bf Colin Rundel$^{1,^\star}$, Roger Bivand$^{2}$, Edzer Pebesma$^{3}$}
\end{center}

\begin{affiliations}
1. Duke University, Department of Statistical Science \\[-2pt]
2. Norwegian School of Economics, Department of Economics \\[-2pt]
3. University of M\"unster, Institute for Geoinformatics \\[-2pt]
$^\star$Contact author: \href{mailto:rundel@gmail.com}{rundel@gmail.com}\\
\end{affiliations}

\keywords Geospatial, GIS, geometry, sp 

\vskip 0.8cm

\pkg{rgeos} is a package that implements functionality for the manipulation and querying of spatial geometries using the Geometry Engine - Open Source (GEOS) \proglang{C} library. This package expands on existing spatial functionality in \proglang{R} through integration with \pkg{sp} spatial classes and transparently replaces \pkg{gpclib} which is encumbered by a licensing agreement allowing only non-commercial use. Additionally, \pkg{rgeos} includes functionality for spatial predicates and topology operations for non-polygon geometries like points, lines, linear rings, and heterogeneous geometry collections. Previously, these operations were not possible within \proglang{R} and required the use of external GIS tools like GRASS, PostGIS, or ArcGIS which significantly complicate spatial workflows.

This talk will discuss the basic usage of this package with a focus on real world use-cases from the R-sig-Geo mailing list. Additionally, we will cover some of the finer details of the GEOS library which will give insight into the most efficient approaches for employing \pkg{rgeos}. Finally, we will discuss future directions for the package with plans for additional features such as spatial indexes using GEOS' \code{STRtree} functionality and the addition of improvements made in GEOS 3.3.0.


\end{document}
