% \iffalse meta-comment -------------------------------------------------------
% Copyright 2015 Matthias Vogelgesang and the LaTeX community. A full list of
% contributors can be found at
%
%     https://github.com/matze/mtheme/graphs/contributors
%
% and the original template was based on the HSRM theme by Benjamin Weiss.
%
% This work is licensed under a Creative Commons Attribution-ShareAlike 4.0
% International License (https://creativecommons.org/licenses/by-sa/4.0/).
% ------------------------------------------------------------------------- \fi
% \iffalse
%<driver> \ProvidesFile{beamerinnerthememetropolis.dtx}
%<*package>
\NeedsTeXFormat{LaTeX2e}
\ProvidesPackage{beamerinnerthememetropolis}
    [2015/06/12 A Modern Beamer Theme]
%</package>
%<driver> \documentclass{ltxdoc}
%<driver> \usepackage{beamerinnerthememetropolis}
%<driver> \begin{document}
%<driver> \DocInput{beamerinnerthememetropolis.dtx}
%<driver> \end{document}
% \fi
% \CheckSum{0}
% \StopEventually{}
% \iffalse
%<*package>
% ------------------------------------------------------------------------- \fi
% \section{Implementation: \textsc{metropolis} inner theme}
%
% A |beamer| inner theme dictates the style of the frame elements traditionally
% set in the ``body'' of each slide. These include:
%
% \begin{itemize}
%   \item title, part, and section pages;
%   \item itemize, enumerate, and description environments;
%   \item block environments including theorems and proofs;
%   \item figures and tables; and
%   \item footnotes and plain text.
% \end{itemize}
%
% \subsection{Title page}
%
% \begin{macro}{title page}
%
%   Template for the title page.
%
%    \begin{macrocode}
\RequirePackage{tikz}
\setbeamertemplate{title page}{
  \begin{minipage}[b][\paperheight]{\textwidth}
%    \end{macrocode}
%
%   If the user has set a |titlegraphic|, we set it in a zero-height box so
%   it doesn't change the position of other elements.
%
%    \begin{macrocode}
    \ifx\inserttitlegraphic\@empty\else{%
      \vbox to 0pt {
        \vspace*{2em}
        \usebeamercolor[fg]{titlegraphic}%
        \inserttitlegraphic%
      }%
      \nointerlineskip%
    }
    \fi
    \vfill%
%    \end{macrocode}
%
%   We set the title and subtitle, but only if they are defined by the user.
%   If |\subtitle| is empty, for example, it won't leave a blank space on the
%   title slide.
%
%    \begin{macrocode}
    \ifx\inserttitle\@empty\else{{%
      \raggedright%
      \linespread{1.0}%
      \usebeamerfont{title}%
      \usebeamercolor[fg]{title}%
      \mthemetitleformat{\inserttitle}%
      \par%
      \vspace*{0.5em}
    }}
    \fi
    \ifx\insertsubtitle\@empty\else{{%
      \usebeamerfont{subtitle}%
      \usebeamercolor[fg]{subtitle}%
      \insertsubtitle%
      \par%
      \vspace*{0.5em}
    }}
    \fi
%    \end{macrocode}
%
%   A horizontal rule (drawn in TikZ) separates the title and subtitle from
%   the author, date, and institution.
%
%    \begin{macrocode}
    \begin{tikzpicture}
      \usebeamercolor{title separator}
      \draw[fg] (0, 0) -- (\textwidth, 0);
    \end{tikzpicture}%
    \par%
    \vspace*{1em}%
%    \end{macrocode}
%
%   Like the title and subtitle, we display the author only when it is defined.
%   But beamer's definition of |\insertauthor| is always nonempty, so we have
%   to test another macro initialized by |\author{...}| to see if the user has
%   defined an author. This solution was suggested by Enrico Gregorio in an
%   answer to \href{https://tex.stackexchange.com/questions/241306/}{this
%   Stack Exchange question}.
%
%    \begin{macrocode}
    \ifx\beamer@shortauthor\@empty\else{{%
      \usebeamerfont{author}%
      \usebeamercolor[fg]{author}%
      \insertauthor%
      \par%
      \vspace*{0.25em}
    }}
    \fi
%    \end{macrocode}
%
%   The date and institute are set after the author, again provided they are
%   nonempty. Note that the default date in \LaTeX{} is |\today|, not |\empty|.
%
%    \begin{macrocode}
    \ifx\insertdate\@empty\else{{%
      \usebeamerfont{date}%
      \usebeamercolor[fg]{date}%
      \insertdate%
      \par%
    }}
    \fi
    \ifx\insertinstitute\@empty\else{{%
      \vspace*{3mm}
      \usebeamerfont{institute}%
      \usebeamercolor[fg]{institute}%
      \insertinstitute%
      \par%
    }}
    \fi
    \vfill
    \vspace*{1mm}
  \end{minipage}
}
%    \end{macrocode}
% \end{macro}%
%
% Normal people should use |\maketitle| or |\titlepage| instead of using the
% |title page| beamer template directly. Beamer already defines these macros,
% but we patch them here to make the title page |[plain]| by default, remove
% |\@thanks|, and ensure the title frame number doesn't count.
%
% \begin{macro}{\maketitle}
% \begin{macro}{\titlepage}
%
%   Inserts the title frame, or causes the current frame to use the
%   |title page| template.
%
%    \begin{macrocode}
\def\maketitle{%
  \ifbeamer@inframe
    \titlepage
  \else
    \frame[plain]{\titlepage}
  \fi
}
\def\titlepage{%
  \usebeamertemplate{title page}
}
%    \end{macrocode}
% \end{macro}
% \end{macro}
%
%
%
% \subsection{Section page}
%
% \begin{macro}{section page}
%
%   Template for the section title slide at the beginning of each section.
%
%    \begin{macrocode}
\setbeamertemplate{section page}{
  \vspace{2em}
  \centering
  \begin{minipage}{22em}
    \usebeamercolor[fg]{section title}
    \usebeamerfont{section title}
    \insertsectionhead\\[-1ex]
    \usebeamertemplate*{progress bar in section page}
  \end{minipage}
  \par
}
\if@noSectionSlide\else%
  \AtBeginSection{
    \ifbeamer@inframe
      \sectionpage
    \else
      \frame[plain,c]{\sectionpage}
    \fi
  }
\fi
%    \end{macrocode}
% \end{macro}
%
% \begin{macro}{progress bar in section page}
%
%   Template for the progress bar displayed by default on the section page.
%   This code is duplicated in large part in the outer theme's template
%   |progress bar in head/foot|.
%
%    \begin{macrocode}
\RequirePackage{calc}
\newlength{\metropolis@progressonsectionpage}
\setbeamertemplate{progress bar in section page}{
  \setlength{\metropolis@progressonsectionpage}{%
    \textwidth * \ratio{\insertframenumber pt}{\inserttotalframenumber pt}%
  }%
  \begin{tikzpicture}
    \draw[bg, fill=bg] (0,0) rectangle (\textwidth, 0.4pt);
    \draw[fg, fill=fg] (0,0) rectangle (\metropolis@progressonsectionpage, 0.4pt);
  \end{tikzpicture}%
}
%    \end{macrocode}
%
%   The above code assumes that |\insertframenumber| is less than or equal to
%   |\inserttotalframenumber|. However, this is not true on the first compile;
%   in the absence of an |.aux| file, |\inserttotalframenumber| defaults to 1.
%   This behaviour could cause fatal errors for long presentations, as
%   |\metropolis@progressonsectionpage| would exceed \TeX's maximum length
%   (16383.99999pt, roughly 5.75 metres or 18.9 feet).
%   To avoid this, we increase the default value for |\inserttotalframenumber|;
%   presentations with over 4000 slides will still break on first compile, but
%   users in that situation likely have deeper problems to solve.
%
%    \begin{macrocode}
\def\inserttotalframenumber{100}
%    \end{macrocode}
% \end{macro}
%
%
%
% \subsection{Block environments}
%
%    \begin{macrocode}
\newlength{\leftrightskip}
\if@beamer@metropolis@blockbg
  \setlength{\leftrightskip}{1ex}
\else
  \setlength{\leftrightskip}{0ex}
\fi
\setbeamertemplate{block begin}{%
  \vspace*{1ex}
  \begin{beamercolorbox}[%
    ht=2.4ex,
    dp=1ex,
    leftskip=\leftrightskip,
    rightskip=\leftrightskip]{block title}
      \usebeamerfont*{block title}\insertblocktitle%
  \end{beamercolorbox}%
  \vspace*{-1pt}
  \usebeamerfont{block body}%
  \begin{beamercolorbox}[%
    dp=1ex,
    leftskip=\leftrightskip,
    rightskip=\leftrightskip,
    vmode]{block body}%
}
\setbeamertemplate{block end}{%
  \end{beamercolorbox}
  \vspace*{0.2ex}
}
%    \end{macrocode}
%
% Alerted block environment
%
%    \begin{macrocode}
\setbeamertemplate{block alerted begin}{%
  \vspace*{1ex}
  \begin{beamercolorbox}[%
    ht=2.4ex,
    dp=1ex,
    leftskip=\leftrightskip,
    rightskip=\leftrightskip]{block title alerted}
      \usebeamerfont*{block title alerted}\insertblocktitle%
  \end{beamercolorbox}%
  \vspace*{-1pt}
  \usebeamerfont{block body alerted}%
  \begin{beamercolorbox}[%
    dp=1ex,
    leftskip=\leftrightskip,
    rightskip=\leftrightskip,
    vmode]{block body}%
}
\setbeamertemplate{block alerted end}{%
  \end{beamercolorbox}
  \vspace*{0.2ex}
}
%    \end{macrocode}
%
% Example block environment
%
%    \begin{macrocode}
\setbeamertemplate{block example begin}{%
  \vspace*{1ex}
  \begin{beamercolorbox}[%
    ht=2.4ex,
    dp=1ex,
    leftskip=\leftrightskip,
    rightskip=\leftrightskip]{block title example}
      \usebeamerfont*{block title example}\insertblocktitle%
  \end{beamercolorbox}%
  \vspace*{-1pt}
  \usebeamerfont{block body example}%
  \begin{beamercolorbox}[%
    dp=1ex,
    leftskip=\leftrightskip,
    rightskip=\leftrightskip,
    vmode]{block body}%
}
\setbeamertemplate{block example end}{%
  \end{beamercolorbox}
  \vspace*{0.2ex}
}
%    \end{macrocode}
%
%
%
% \subsection{Itemize/enumerate environments}
%    \begin{macrocode}
\setlength{\leftmargini}{1em}
\setlength{\leftmarginii}{1em}
\setlength{\leftmarginiii}{1em}
\setbeamertemplate{itemize item}{\textbullet}
\setbeamertemplate{itemize subitem}{\textbullet}
\setbeamertemplate{itemize subsubitem}{\textbullet}
%    \end{macrocode}
%
% \subsection{Figures and tables}
%    \begin{macrocode}
\setbeamertemplate{caption label separator}{: }
\setbeamertemplate{caption}[numbered]
%    \end{macrocode}
%
% \subsection{Footnotes}
%    \begin{macrocode}
\setbeamertemplate{footnote}{%
  \parindent 0em\noindent%
  \raggedright
  \usebeamercolor{footnote}\hbox to 0.8em{\hfil\insertfootnotemark}\insertfootnotetext\par%
}
%    \end{macrocode}
%
% \subsection{General text}
%    \begin{macrocode}
\mode<all>
\setlength{\parskip}{0.5em}
\linespread{1.15}
%    \end{macrocode}
%
%
%
% \iffalse
%</package>
% \fi
% \Finale
\endinput
